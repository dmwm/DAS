\documentclass[a4paper,11pt,draft]{article}
\special{papersize=210mm,297mm}

% twocolumn
%\usepackage{fullpage} 
\usepackage[cm]{fullpage}
%\usepackage[margin=0.5in]{geometry}
%\usepackage{times}
\usepackage{color}
%\usepackage[small,compact]{titlesec}

\bibliographystyle{alpha} 
 
% usually first paragraph is NOT indented, so this is commented out: 
% \usepackage{indentfirst} 
 
\begin{document}
 
% Article top matter
\title{Master thesis problem statement: Keyword search over heterogeneous sources} %\LaTeX is a macro for printing the Latex logo
\author{Vidmantas Zemleris}  %\texttt formats the text to a typewriter style font
\date{\today}  %\today is replaced with the current date
 
%\maketitle

%
 \centerline{\Large \bf Searching heterogeneous data-sources: Master thesis problem statement} %% Paper title
 \medskip
 
 \centerline{Vidmantas Zemleris, \today}  %% Author name(s)
 \medskip
 


 
\section{Introduction}
At large scientific collaborations like the CMS Experiment at CERN's LHC that includes more than 3000 collaborators data usually resides on a fair number of autonomous and heterogeneous proprietary systems each serving it's own purpose
\footnote{For instance, at CERN,  due to many reasons (e.g. research and need of freedom, politics of institutes involved) software projects usually evolve in independent fashion resulting in fair number of proprietary systems\cite{Koch00CERN}. Further high turnover makes it harder extending these systems}. As data stored on one system may be related to data residing on other systems%
	\footnote{For example, datasets containing physics events are registered at DBS, while the physical location of files is tracked by Phedex which also takes care of their transfers within the worldwide grid storage}%
, users are in need of a centralized and easy-to-use solution for locating and combining data from all these multiple services.

Using a highly structured language like SQL is problematic because users need to know not only the language but also where to find the information and also lots of technical details like schema. A data integration system based on simple structured queries is already in place. Still {\color{red}various improvements including support} for less restricted keyword queries and improvements to system's usability and performance still have to be researched.

% {\color{red}A remarkable opportunity we have is possibility to test ideas on a sample of ~3000 people working at CMS.}


\section{Case study: the CMS Experiment at CERN}
Users' information need may vary greatly depending on their role, however most of the time they are interested in locating full set of entities matching some selection criteria, e.g.:
%\footnote{in contrary to exploratory approach where  a subset of best matches is enough}
   \begin{itemize}
  		\item find all \textit{files} from \textit{dataset(s)} matching wild-card query each containing some of the 'interesting' \textit{runs} from a list provided (Release validation teams)
         \item find (all) \textit{datasets} related some specific physics phenomena\footnote{In case of dataset this data is present in filename or run} together with conditions describing how this data was recorded by detector or simulated which are present in separate autonomous system than the datasets (Physicists)
         % TODO:is this a good example? conditions is a separate query now...
         \item find all \textit{datasets} matching some pattern stored at a given \textit{site} (filtering on entities stored on separate services)
   \end{itemize}                		
% In such dynamic environment it is preferred to choose local-as-view approach for information integration instead of creating a global schemata\cite{Koch00CERN}.

For more use-cases of data retrieval at CMS Experiment see \cite{CMS_data08}.


%Following a similar approach a 
\subsection*{The Data Aggregation System}

The Data Aggregation System (DAS)\cite{Kuznetsov2010, Kuznetsov2011} was created which allows integrated access to a number of proprietary data-sources by processing user's queries on demand - it queries the data-sources, merges the results, and caches them for subsequent uses. 

Currently the queries specify what entity the user is interested in (dataset, file, etc) and provide selection criteria (attribute=value, name BETWEEN [v1, v2]) operators. The combined query results could be later 'pipped' for further filtering and aggregation (min, avg, etc), e.g.:

{\footnotesize 
\begin{verbatim}
dataset=*RelVal* | grep dataset.nevents >1000 | avg(dataset.size), median(dataset.size)
\end{verbatim}
}

The query above would return average and median datasets sizes  of ones containing  \textit{RelVal} in their name having more than 1000 events.

Queries could be run either from web browser or through  command line interface where the results could  be fed into another application (e.g. program doing physics analysis or automatic software release validation).


% \textbf{\color{red}Examples of more general uses: Digital libraries, Stock markets, [ Weather, Flight costs]}

\section{Problem statement}

\textbf{\color{red}DO WE also have proprietary big databases that could be useful to be searched, but the standard integration work would be to heavy (or sub-parts of existing systems that are not covered by APIs, and could be accessible through DBs directly -- these potentially more expensive queries could processed with lower priority)}

\subsection{Ease of use: Query Language over the mediated schema}
	Even a very simple structured query language that also contain entity names over the mediated schema may seem {\color{red}hard} to learn, especially in the beginning.
   On the other hand, at CERN the names in  the mediated schema are referring to real-world entities that {\color{red}are fairly consistently named} (even though there may exist slight differences in their naming on different data-sources). 
   So some sort of guiding helping user to build the query shall be useful.
        
	Further a minimum set of search predicates is imposed by APIs (mainly because of performance reasons) and user has to be at least informed what is he expected to provide.
                
                
Also to consider:
\begin{itemize}         
       \item input ambiguity, typos
       \item {\color{red} what about ranked search?}
\end{itemize}

\subsection{Performance}
\subsubsection{Handling distributed search efficiently}

\subsubsection{Other performance issues}
	     \begin{itemize}
	     					\item  {\color{red}maybe we could have prioritization in general, the heavier your query is the more you have to wait in favour of  the light queries}: 
	     					(we could have some sort of query cost evaluation based on history or manually predefined scores per API)
	     					
	                		\item more intelligent caching needed:
	                			\begin{itemize}
	                			\item given an entity received from provider, determining if it is useful to cache for long-term
	                			\item \textbf{\color{red}At what level are we caching now? query or individual query results?}
	                			\item could make use of even older information warning user
	                			\item could also have different validity dates for certain fields. if certain field is not explicitly requested, even a very old cache could be used.
	                			\end{itemize}
	                	  
	                		\item query rewriting then used with grep=smf then the same item is available as selection key. check how many instances.
	                		\item pre-fetching common (sub-)queries: determining manually and/or automatically
	                		\item scale testing - if we are storing lots of historical info. MongoDB is not so performant if DB cant fit in memory. One of well known solutions is installing SSD.
                        \item ? with current API's for aggregation service has to fetch ALL records not just the first page..
                        \item ??? do we have cases then multiple APIs of the same service could be called for the same data, can we choose just one???
                        \item ? showing partial results
        \end{itemize}



\section{Preliminary Literature review}

% Overview: keyword search is good for non-structured documents, it is not [as] effective with structured sources\cite{Levy96}, therefore the keyword search on relational databases approach is good then no specific APIs exist and it's there are no resources to do \textbf{a full integration} .

\textbf{\color{red}TODO: Overview; evaluate performance}


\subsection{Keyword search over DBs}
The problem of Keyword search over Relational Databases (or also semi-structured sources like XML) has received a significant attention by the research community over the last decade. 

The basic approach would first build an inverted index on database tables (usually only text columns). Then after finding all occurrences of the keywords, would try to construct join paths (based on Foreign keys) that would unite tuples containing the keywords.

% In addition to many papers the PhD dissertations \cite{PhD_2011, PhD_2012} describe the approaches in details including performance optimization details (e.g. generating materialized views), etc. 
% 
A number of variations exist:  returning only the Top-k results by some ranking function, SQAK and SODA would allow generating even more complex queries including aggregations.



\subsubsection*{Ranking Query Templates based on keyword query}
A simple way to access relational database could  be through a set of predefined named query templates (SQL with selection parameters or operators still to be specified) exposed to a user as a Form that the user has to fill in.

\cite{forms_kws} proposes alternative approach for processing keyword queries over relational databases: given a keyword query, instead of returning database tuples one could rank query forms that best matches the query for user to choose the right one (if they are properly named this is fairly easy). The ranking is based on checking matching of keywords to table names in templates and to column values (could be implemented with inverted index).\textbf{\color{red}TODO: more detailed and our limitations (after reading keyword cleaning)}.

An interesting feature of this approach is that a Query Template is functionally similar to any autonomous web service (which given the parameters would in turn execute that query on its database).  In case of the Data Aggregation System, a user after entering a keyword query could be provided with a ranked list of structured queries (attribute=value) that could be processed given data source constraints (e.g. parameters required) and if needed he could refine his search (e.g. provide more parameters).


\subsubsection*{Keyword query cleaning}
{\color{red} From first sight looks fairly cool. Build index with some powerful fulltext search engine (e.g. Xapian) and use it to generate mapping to structured queries}




\subsubsection*{Meta-data approach}

With a goal to bridge the increasing gap between high-level (conceptual, business) and low level (physical) representations of data, researchers from \textit{ETHZ} have been investigating Generation of SQL for Business users at \textit{Credit Suisse}.  For converting natural language queries 
% (that in addition to keyword search could convey some semantic structure)
 into SQL statements, in addition to what used by earlier approaches they used meta-data describing the schema (at multiple representation levels) and the domain (ontologies) and some natural language processing.

Even on a large data-warehouse of ~220GB data with a complex schema of 400+ tables they reported that if good meta-data is available, generating even quite complex SQL  (n-way joins with aggregations, etc) is quite feasible for computer. That would making it 'much easier for business users to interactively explore highly-complex data warehouses' \cite[p.932]{ethz2012}. 


\subsubsection*{Then there is no access to index data terms}
{\color{red} TODO: Describe Keymantic\cite{Keymantic10}. This is a suboptimal solution as indexing terms improves results. }


%\subsubsection*{Automatic mapping between distributed services}
%{\color{red} TODO: an interesting approach but not for CERN...}



%\subsubsection*{\color{red}Question Answering}


% See Appendix #1, for evaluation of they Demo system.



\subsection{Searching 'Deep Web' and Heterogeneous web services}
{\color{red}There doesn't seem to be much of recent papers going towards this direction (except from specific search systems like flight fare comparisons), however these are based on structured arguments.} 

In 1990s there were was much research on search based service integration systems, for instance \textit{Information Manifold} following Local-as-View approach and \textit{TSIMMIS} developed at Stanford following Global-as-View approach. 

The \textit{Information Manifold} is quite resembling to DAS {\color{red},  so it could be useful checking it's papers. TODO: finish}

{\color{red}Check deep web (also recent research exist)}
{\color{red}Performance: Answering Queries using Views}


\begin{verbatim}
Some other approaches could include: 

- creating virtual documents offline by joining tables, for instantaneous search results by employing IR techniques
(Indexing Relational Database Content Offline for Efficient Keyword-Based Search, 2003)


- (Efficient Keyword Search Across Heterogeneous Relational Databases, 2007)
 : combines schema matching and structure discovery techniques to find approximate foreign-key joins across heterogeneous databases
\end{verbatim}


\section{Possible approaches}


\subsection{Ease of use}
\begin{verbatim}
- javascript interpreter
    : field + description -- dropdown searcheable list
    
- if entered keyword query suggest a structured query
	based on 'Query forms' and 'Keyword cleaning' approaches
	--> evaluate different approaches (HMM, etc) and implementations
	optionally with description in natural language
- select entity (e.g. dataset) --> give all available keys from forms, however there are still combinations which are enforced
\end{verbatim}
%(could use Natural Language Generation)

\subsection{Performance}
\subsubsection{Continuous view maintenance at Data providers with large DBs}
Use either \textit{materialized refresh fast views with query rewriting} (Oracle; completely transparent for proprietary apps)\cite{Oracle11}
 or some other continuous view maintenance tool (e.g. DBToaster\footnote{http://www.dbtoaster.org that is being developed at EPFL}) to improve performance of heavy queries containing joins and/or aggregations.

\subsection{Integrating distributed information efficiently}
Bloom-join (which could be quite transparent and implemented even on DB side [pure SQL is possible for MySQL, to check for Oracle]- take a query and bit-vector as parameter) , lazy pagination (and order required for aggregation) - this is not yet supported by any of the data-service APIs

%\newline
integration at DAS level. (at source DB could be more performant a little)

\subsection{More intelligent caching}
Some of the data entities instances changing very rarely, for example in DBS system old datasets would never change, while new ones are constantly added (still some of their attributes may change, in this example validity).

Cached copy may be shown before hand while up to date results could be retrieved on user's request.

\textbf{Decide if to cache or not...}


An automatic change rate prediction could be useful to efficiently balance between caching and retrieving results.

\begin{verbatim}
Also: Caching of intermediary results, not only the queries
\end{verbatim}

\section{Work status}
TODO

\begin{verbatim}
- obtained DB copy of biggest data provider DBS (currently 80 GB + 200 GB indexes)
- preliminary literature review (to be continued deeper + waiting for book arrival: Principles of Data Integration)

--- some analysis of DAS logs


Upcomming Work items:
- couple of fairly simple prototypes of UI/access patters for simpler DAS querying
- check on performance improvements after creation of materialized view(s)
- interview (more) DAS users
\end{verbatim}

% ---  Bibliography
\bibliographystyle{unsrt}
\addcontentsline{toc}{chapter}{Bibliography}

%\nocite{*}
\thispagestyle{empty}
\bibliography{refs}


\end{document}
