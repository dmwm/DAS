\documentclass[a4paper,11pt]{article}
% ,draft
\special{papersize=210mm,297mm}

% twocolumn
%\usepackage{fullpage} 
\usepackage[cm]{fullpage}

%\addtolength{\topmargin}{-10mm}
%\addtolength{\topskip}{-10mm}

%\usepackage[margin=0.5in]{geometry}
%\usepackage{sans}
%\usepackage{lmodern}
\usepackage{paralist}
\usepackage{color}

% enable Hyperlinks
\usepackage{hyperref}
% fancy verbatim
\usepackage{fancyvrb}
\usepackage{appendix}
\usepackage[usenames,dvipsnames]{xcolor}


%\usepackage{alltt}

%\usepackage[small,compact]{titlesec}

\bibliographystyle{alpha} 
 
% usually first paragraph is NOT indented, so this is commented out: 
% \usepackage{indentfirst} 
 
\begin{document}
%\addtolength{\bottommargin}{-10mm}
\addtolength{\textheight}{10mm}

 
% Article top matter
\title{Master thesis problem statement: Keyword search over heterogeneous sources} %\LaTeX is a macro for printing the Latex logo
\author{Vidmantas Zemleris}  %\texttt formats the text to a typewriter style font
\date{\today}  %\today is replaced with the current date
 
%\maketitle

%
 \centerline{\Large \bf Searching heterogeneous data-sources: Master thesis problem statement} %% Paper title
 \medskip
 
 \centerline{Vidmantas Zemleris, \today}  %% Author name(s)
 \medskip
 


 
\section{Introduction}
At large scientific collaborations like the CMS Experiment at CERN's LHC that includes more than 3000 collaborators data usually resides on a fair number of autonomous and heterogeneous proprietary systems each serving it's own purpose
\footnote{For instance, at CERN,  due to many reasons (e.g. research-orientedness and need of freedom, politics of institutes involved) software projects usually evolve in independent fashion resulting in fair number of proprietary systems\cite{Koch00CERN}. Further high turnover makes it harder to extend these systems}. As data stored on one system may be related to data residing on others%
	\footnote{For example, datasets containing physics events are registered at DBS, while the physical location of files is tracked by Phedex which also takes care of their transfers within the grid storage worldwide}%
, users are in need of a centralized and easy-to-use solution for locating and combining data from all these multiple services.

Using a highly structured language like SQL is problematic because users need to know not only the language but also where to find the information and also lots of technical details like schema. An enterprise information integration system based on simple structured queries is already in place. Various improvements  still have to be researched: adding support for more complex queries and for less restricted keyword search, improvements to system's usability and performance.

% {\color{red}A remarkable opportunity we have is possibility to test ideas on a sample of ~3000 people working at CMS.}


\section{Case study: the CMS Experiment at CERN}
Users' information need may vary greatly depending on their role, however most of the time they are interested in locating a \textit{full set} of entities matching some selection criteria, e.g.:
%\footnote{in contrary to exploratory approach where  a subset of best matches is enough}
   \begin{itemize}
  		\item find all \textit{files} from \textit{dataset(s)} matching wild-card query each containing some of the 'interesting' \textit{runs} from a list provided (Release validation teams)
         \item find all \textit{datasets} matching some pattern stored at a given \textit{site} 
         	  \begin{small}(filtering attributes from separate services)\end{small}
         \item find (all) \textit{datasets} related to some specific physics phenomena\footnote{In case of dataset, this data is present in the \textit{filename} or in the \textit{run} entity} together with conditions describing how this data was recorded by detector or simulated which are present in another autonomous system than the datasets (Physicists)
         % TODO:is this a good example? conditions is a separate query now...
   \end{itemize}                		
% In such dynamic environment it is preferred to choose local-as-view approach for information integration instead of creating a global schemata\cite{Koch00CERN}.

For more use-cases of data retrieval at CMS Experiment see \cite{CMS_data08}.

%Following a similar approach a 
\subsection*{The Data Aggregation System}

The \textit{CMS Data Aggregation System (DAS)}\cite{Kuznetsov2010, Kuznetsov2011} was created which allows integrated access to a number of proprietary data-sources by processing user's queries on demand - it determines  data-sources are able to answer\footnote{This is done by a mapping between flat mediated schema entities ('DAS keys') into web-service methods and their arguments. Then system queries all services that could provide a result of expected type with given parameters}, queries them, merges the results and caches them for subsequent use. DAS uses \textit{Boolean retrieval model} as users are often interested in retrieving ALL the items matching their query.

Currently the queries specify what entity the user is interested in (dataset, file, etc) and provide selection criteria (attribute=value, name BETWEEN [v1, v2]). The combined query results could be later 'piped' for further filtering, sorting or aggregation (min, max, avg, sum, count, median), e.g.:

{\small 
\begin{Verbatim}[commandchars=\\\{\}]
\footnotesize# returns average and median dataset sizes of ones containing \textit{RelVal} in their name having more than 1000 events.
dataset=*RelVal* | grep dataset.nevents >1000 | avg(dataset.size), median(dataset.size)
\end{Verbatim}
}

Queries are executed either from web browser or through a command line interface where the results could be fed into another application (e.g. program doing physics analysis or automatic software release validation).
% \textbf{\color{red}Examples of more general uses: Digital libraries, Stock markets, [ Weather, Flight costs]}

\section{Problem statement}
\subsection{User Interface and ease of use}

For an IR system with a wide variety of users, it important to provide an easy to use interface with fairly flat learning curve, while not loosing support of fairly complex queries.
%
	Even a simple structured query language plus entity names over the mediated schema  at first may seem hard to learn%
	\footnote{Actually, the names in DAS mediated schema refer to entities fairly consistently named in the real-world.}. Therefore, it shall be useful to guide new users interactively through the process of building the query.
%   
   Supporting non-structured keyword queries is also worth investigation as quite many users reported missing this Google-like search experience.
        
        
\subsection{Supporting complex queries}
Currently DAS can only process queries that could be directly mapped to data providers' APIs, but not the queries where results of one API need to be fed into another APIs 
	%\footnote{this is partially because of performance reasons {\color{red}and the fact that DAS is mainly schema-less (the mediated schema is only in terms of entities, their mappings to APIs and API parameters)}}%
	\footnote{Except a couple of manually hard-coded "views" in a virtual \textit{"combined"} data provider}%
. %In such case, the user  is forced to write some code to submit and process the sub-queries himself. 
Further, to be flexible DAS only keeps a list of 'mediated entities' and how they could be retrieved, but do not enforce any predefined schema (i.e. entity fields/attributes) nor exposes it\footnote{%
	Actually the fields of each API  (an entity could be mapped to a number of these) could be inferred 
	from past queries or by predefined 'bootstrap' queries. Currently the field lists may differ slightly depending on the parameters}
before querying for some particular entity(-ies). 
%
This makes it fairly hard to execute complex queries  without a priori knowledge.
User has to know what queries and how have to be combined (depends on the mediated schema and what APIs are available), and has either to combine the results by hand or to write a program doing that.
%
To be researched:
	\begin{compactitem}
		\item how should these queries be entered by users?

		\item could we use current source descriptions that only define API result type and API parameters as entities of mediated schema to select meaningful and sufficiently efficient composition of services?
	\end{compactitem}
	
%+\item {\color{green} Build series of queries which will answer the given user question, e.g. user asks {\it I want to find files which contains runs taken with magnetic field 4T for given dataset at site X}. This should be decomposed somehow (subject to research) into series of quries, e.g. Find datasets at site X; find files for given dataset list; find runs for found files; filter runs with magnetic field 4T}
       
\subsection{Performance}

DAS is based on \textit{Virtual Integration} where data is left at the sources allowing data to be volatile, e.g. new records gathered or existing ones updated, service's schema changed. Still large amounts of the data are fairly static. 
% %the providers may impose some constraints (e.g. only certain queries allowed).}
As APIs of some data providers' are comparatively slow\footnote{This is partially because of large amounts of data stored at the providers (order of ~1TB per year in total) and as their main goal is supporting data taking from the CMS detector their systems were not optimzed for exploratory queries} it is important to balance between quickly returning items from local cache and getting fresh results. Another approach we are taking is spotting the performance issues at the services (e.g. at DBS some queries requires joining many very large tables, while most of their existing data stays static).
%
Issues to be addressed includes:
\begin{compactitem}
	     				\item At the moment all queries are put into one pool and has to wait until some threads are available. If a couple of heavy queries were submitted earlier, even light queries would have to wait long. Explore more advanced Query prioritization 
	     					(e.g. we could have a query cost model)
						%  based on history or predefined scores per API
	     					
                		\item Currently result items are cached for a fairly short period of time (5min-1h) and then completely discarded, however many entities are not changing that often - Explore more intelligent caching
	                	  
	                		% TODO: \item query rewriting then used with grep=smf then the same item is available as selection key. check how many instances.
	                		

					\item Since DAS does aggregation across multiple data providers, given their current APIs (with no \textit{ordering} and \textit{paging} of results), DAS has to fetch ALL records matching the query instead of only the first page\footnote{Although it would be possible to show partial results for data coming from a single data-source (still loading in background, or the providers' APIs would have to be modified)}.		
					\item Efficient distributed search: as filtering criteria may reside on two or more autonomous sources, given the current APIs much more items than in the final result may need to be fetched from each source. Items that do not meet all selection criteria could be filtered out only afterwards in DAS. To improve the performance, data provider's APIs for the most used queries may need to be redesigned.
                        % \item {\color{red}e.g. displaying intermediary results of aggregation while still calculating with statistical bounds -- however statistical bounds are much meaningful only if items are well snuffled/randomized -- people could stop the query if they see a non-sense}
	              %\item {\color{red}scale testing - if we are storing lots of historical info. MongoDB is not so performant if DB cant fit in memory. One of well known solutions is installing SSD.                        }
\end{compactitem}

\subsection{\color{gray}Generic connector accessing relational databases}

{\color{gray}A generic connector accessing relational databases with minimal human effort could be useful for integrating proprietary systems that were not yet integrated when there's no resources to build the APIs. A possible use-case could be the \textit{Prep} database, but this still has to be discussed. 
%
%Depending on the needs it could be either of exploratory or API/Query Forms based approach, where fake APIs are defined in terms of SQL queries (or even better in a simplified form that would generate SQL by taking schema into account).
There was a project\footnote{\url{https://github.com/vkuznet/PyQueryBuilder}}   at CERN which tried to generalize DBS-Query Language for any database that could be used as DB side mediator.
}
% \color{green} Indeed, an dthat was the topic of research for another DAS student, Liang Dong. The idea was to generalize DBS-QL to any DB back-end and provide common package which will do a job. The work is done over here . Feel free to check it out and we can discuss it more.}


\section{Preliminary Literature review}

% Overview: keyword search is good for non-structured documents, it is not [as] effective with structured sources\cite{Levy96}, therefore the keyword search on relational databases approach is good then no specific APIs exist and it's there are no resources to do \textbf{a full integration} .

\subsection{Overview}


The problem of keyword search over \textit{relational and other structured databases} is widely covered,  including even keyword search directly over distributed relational databases\cite[ch.16]{principles_data_integration} and some approaches for automatic schema matching has been discussed also. Such search is however very ambiguous and has quite large search space over the possible join paths, therefore most approaches rely on returning the top-k results.

Regarding the Boolean retrieval that we are the most interested, keyword query could be translated into list of best structured queries (e.g. SODA system proposing SQL over complex schemas, while Query Forms would propose mappings from keywords to parameters in SQL templates, Keymantic tries to achieve this without having access to Data itself). 
% TODO: Keymantic would propose SQL queries knowing only schema but no index

%  TODO: Further, {\color{red} in theory,} Google's Deep-web crawl techniques initially designed for accessing content hidden behind html forms could be used to effectively index the contents even of  data providers that do not provide a specific interface to do so.

{\color{red}Current research trends seem to fall on methods of automatic integration (e.g. statistical mediator), and pay-as-you-go integration. (TODO: see future chapter of \cite{principles_data_integration})}

%\textbf{\color{red}TODO: Overview; evaluate performance}

\subsection{Searching 'Deep Web' and web services}

In 1990s there were was much research on search based service integration systems, for instance \textit{Information Manifold} quite resembling DAS {\color{red},  so it could be useful checking it's papers.}.
% following Local-as-View approach
%and \textit{TSIMMIS} developed at Stanford following Global-as-View approach. The \textit{Information Manifold} is 

\subsubsection*{No access to index data terms}

\cite{Keymantic10, semantics_without_access} explores the case then there is no possibility to index the data terms, e.g. then a DB is behind a wrapper (e.g. accessible only through a \textit{Web form} in “Hidden Web” or \textit{a web-service}) then crawling is generally not possible.

In Keymantic\cite{Keymantic10} a 'keyword query is processed as follows: First, all keywords that correspond to metadata items (e.g., field names) are extracted. The remaining
keywords are considered as possible input fields. Second, the likelihood of a remaining keyword to a metadata item is computed in order to rank different options to execute the keyword query on the
“Hidden Web” database'\cite[p.942]{ethz2012}.


\subsubsection*{Deep web search at Google}
[multiple papers from Google] discusses various ways for implementing data integration in terms of large-scale search engine (Google): Virtual integration vs. Surfacing. They also present ways for integrating systems without human intervention through use of statistical 'mediator'.

There two approaches to web scale search for deep-web (Google mostly cares about web forms):

\textit{Runtime query reformulation} - 'leaves data at the sources and routes queries to appropriate services'\cite[p. 1]{webscale_paygo} 

\textit{Deep-web surfacing} - tries to add content from the deep-web into search index. There are algorithms which allow to iteratively choose input parameters to forms to surface a considerable part of the 'hidden' data without large overhead%
	\footnote{i.e. if choosing parameters in not smart way, a web form with just a couple of free inputs (or even dropdowns that's easier case), could yield as many results as a cross product of all input combinations.}.

\textit{PayGo approach}\cite{webscale_paygo}  - With this approach, 'a system starts with
very few (or inaccurate) semantic mappings and these mappings
are improved over time as deemed necessary'.
% TODO: cite{bootstraping}
there is NO single mediated schema over which users pose queries. Queries are routed to to the relevant sources with help statistical methods that are used to model uncertainty at all levels: queries, mappings and underlying data. 

%{\color{red} TODO: Describe Keymantic\cite{Keymantic10}. This is a very suboptimal solution as indexing terms improves results (mention evaluation from SODA paper). It works only then keywords map entity names. some hybrid approaches: 
%index if exists, regexp, some string similarity measure based on historical data (e.g. even edit-distance would work for many items, like site)}



\subsection{Keyword search over relational Database}
The problem of Keyword search over Relational Databases (or also semi-structured sources like XML) has received a significant attention by the research community over the last decade. 

The basic approach would first build an inverted index on database tables (usually only text columns). Then after finding all occurrences of the keywords, would try to construct join paths (based on Foreign keys) that would unite tuples containing the keywords.

% In addition to many papers the PhD dissertations \cite{PhD_2011, PhD_2012} describe the approaches in details including performance optimization details (e.g. generating materialized views), etc. 
% 
A number of problem variations exist:  returning only the ranked Top-k results vs. returning ranked list of possible queries, while some systems would even allow generating more complex queries including aggregations, etc (SQAK, SODA\cite{ethz2012}).

\subsubsection*{Ranking Query Templates based on keyword query}
A simple way to access relational database could  be through a set of predefined named query templates (SQL with selection parameters or operators still to be specified) exposed to a user as a Form that the user has to fill in.

\cite{forms_kws} proposes alternative approach for processing keyword queries over relational databases: given a keyword query, instead of returning database tuples one could rank query forms that best matches the query for user to choose the right one (if they are properly named this is fairly easy). The ranking is based on checking matching of keywords to table names in templates and to column values (could be implemented with inverted index).\textbf{\color{red}TODO: more detailed and our limitations (after reading keyword cleaning)}.

An interesting feature of this approach is that a Query Template is functionally similar to any autonomous web service (which given the parameters would in turn execute that query on its database).  In case of the Data Aggregation System, a user after entering a keyword query could be provided with a ranked list of structured queries (attribute=value) that could be processed given data source constraints (e.g. parameters required) and if needed he could refine his search (e.g. provide more parameters).


\subsubsection*{Keyword query cleaning}
Keyword queries are often ambiguous, may contain misspellings or multiple keywords that refer to the same attribute value,  therefore \cite{kw_cleaning} suggested to perform query cleaning before proceeding to subsequent more computationally expensive steps (e.g. exploring all the possible join paths).

Further employing some machine learning method like HMM\cite{kw_cleaning_hmm} would allow to incorporate user's feedback (even the fact that user has chosen n-th result as a query to be executed is a good clue).


\subsubsection*{SODA: Meta-data approach}

With a goal to bridge the increasing gap between high-level (conceptual, business) and low level (physical) representations of data, researchers from \textit{ETHZ} have been investigating Generation of SQL for Business users over a very complex data warehouse at \textit{Credit Suisse}.  For converting natural language queries 
% (that in addition to keyword search could convey some semantic structure)
 into SQL statements, in addition to what used by earlier approaches they used meta-data describing the schema at both physical, conceptual and logical levels extended with DBpedia (for synonyms, etc) and domain ontologies (to capture business concepts like 'wealthy customer').
%and some natural language processing

Even on a large data-warehouse of ~220GB data with a complex schema of 400+ tables they reported that if good meta-data is available, generating even fairly complex SQL  (e.g. n-way joins with aggregations) is quite feasible for a computer. That would make it 'much easier for business users to interactively explore highly-complex data warehouses' \cite[p.932]{ethz2012}. The users also reported system's potential a) for analysing the schema and learning patterns about it and b) as tool to help documenting legacy systems.


\subsection{Keyword search: integration on demand}
{\color{red}TODO?: \cite[ch.16]{principles_data_integration}}




%\subsubsection*{Automatic mapping between distributed services}
%{\color{red} TODO: an interesting approach but not for CERN...}



%\subsubsection*{\color{red}Question Answering}


% See Appendix #1, for evaluation of they Demo system.



% {\color{red}There doesn't seem to be much of recent papers going towards this direction (except from specific search systems like flight fare comparisons), however these are based on structured arguments.} 

% Some other approaches could include: - creating virtual documents offline by joining tables, for instantaneous search results by employing IR techniques
% (Indexing Relational Database Content Offline for Efficient Keyword-Based Search, 2003)

% TODO: (Efficient Keyword Search Across Heterogeneous Relational Databases, 2007) : combines schema matching and structure discovery techniques to find approximate foreign-key joins across heterogeneous databases



\section{Proposed solutions}
\subsection{Ease of use}

\begin{itemize}
\item Interactive guidance on how to compose queries to ease the learning (could also include some examples of popular queries):
\begin{enumerate}
\item What entity are you searching for?
\item How could you identify it (i.e. what do you know)?
\item After seeing first results: What do you want to do with the results? (list items; select only specific columns; do aggregation; use in command line/another program)
\end{enumerate}

\item (?) JavaScript-based query interpreter to ease query writing: could suggest search attribute and entity attribute names. also taking into account some possible ambiguous namings (auto completion)


\item Keyword query --> ranked list of matching structured queries:
	\begin{itemize}
	\item map keywords to entity name and selection parameters (quite resembling 'Query forms' and 'Keyword cleaning' approaches)
	\item
		{\color{red} Inverted index could be built with help of some good full-text search engine (e.g. Xapian\footnote{http://xapian.org/}) and used to generate mapping to structured queries. If no match found (e.g. new entry not yet in our cache) some pattern matching could be used to try to corresponding guess search attribute}
	\item ranking could be based on: how closely keywords are matching some or all required parameters, popularity of certain query and users feedback
	
	\item {\color{red}	--> evaluate different approaches (HMM, etc); could some natural language processing help?}
	\item special attention to \textbf{wilcard} queries
	
	\end{itemize}
\end{itemize}


\subsection{Performance}
\subsubsection{Continuous view maintenance at Data providers with large DBs}
Use either \textit{materialized refresh fast views with query rewriting} (Oracle; completely transparent for proprietary apps)\cite{Oracle11}
 or some other continuous view maintenance tool (e.g. DBToaster\footnote{http://www.dbtoaster.org that is being developed at EPFL}) to improve performance of heavy queries containing joins and/or aggregations.



\subsubsection{More intelligent caching}
Some of the data entities instances changing very rarely, for example in DBS system old datasets would never change, while new ones are constantly added (still some of their attributes may change, in this example validity).

Cached copy may be shown before hand while up to date results could be retrieved on user's request.

An automatic change rate prediction could be useful to efficiently balance between caching and retrieving results.

Also consider:
                			\begin{itemize}
                			\item given an entity received from provider, determining if it is useful to cache for long-term
%	                			\item \textbf{\color{red}At what level are we caching now? query or individual query results?}
                			\item could cache even \textit{expired} data, but warning user
                			\item could also have different validity dates for certain fields. if no volatile fields are not explicitly requested, even a very old cache could be used. 
                			\item Can we automatically figure which fields are static and which are changing?
                			\item pre-fetching common (sub-)queries: determining manually and/or automatically
                			
                			\item \textbf{Deciding if to cache or not...}

                			\end{itemize}



\subsection{Integrating distributed information efficiently}
Bloom-join (which could be quite transparent and implemented even on DB side [pure SQL is possible for MySQL, to check for Oracle]- take a query and bit-vector as parameter) , lazy pagination (and order required for aggregation) - this is not yet supported by any of the data-service APIs

%\newline
% integration at DAS level. (at source DB could be more performant a little)





\newpage
\section{Work status}
{\begin{footnotesize}
\begin{verbatim}
- set-up of development environment
- preliminary literature review (to be continued deeper)
- initial analysis of DAS logs
- obtained DB copy of biggest data provider DBS (currently 80 GB + 200 GB indexes)
- some work done on solving wildcard query restrictions

Upcomming Work items:
- benchmark data provider's performance (per API based on historical queries)
- couple of fairly simple prototypes of UI/access patters for simpler DAS querying
- check performance improvements after creation of materialized view(s)
- look into possible implementations for combined queries

\end{verbatim}
\end{footnotesize}
}
% ---  Bibliography
\bibliographystyle{unsrt}
\addcontentsline{toc}{chapter}{Bibliography}

%\nocite{*}
\thispagestyle{empty}
\begin{small}
\bibliography{refs}
\end{small}

\pagebreak
\begin{appendix}
\section{Project Milestones}

Initially written up by Valentin (technical lead of the project from US):

\begin{footnotesize}
\begin{Verbatim}[commandchars=\\\{\}]
Sep:
   - familiarize with techniques, research activities in the field
   - setup all tools and learn code to do actual work
   - describe and put on paper CMS use case
   - propose several approaches to target the project
Oct:
   - pick-up couple of approaches for development
   - develop initial prototype in high-level
       - outline workflow
   - math proof of choosen approaches
   - present results at EPFL & CMS DMWM team
Nov:
   - made a choice among target approach and start working on real prototype
   - it would be nice to see math foundations in place, including 
     inital benchmarks, time estimates, stats, etc.
   - mid-term write-up document about current achievements, 
     description of choosen approach and its implementation milestones
   - at this stage work with couple of data-providers (or their limited scope) 
     to demonstrate capabilties of the chosen approach
   - I do not expect at this stage any optimizations, but would like to hear 
     where those can be done, including strategies, technical improvements, etc.
   - do not limit yourself to CMS use case, explore possibility to apply
       choosen approach outside of CMS field
     - a toy prototype with non CMS data-providers is desired to have
            - demonstrate and discuss if it can be applied elsewhere
   - present results at EPFL & CMS DMWM team
   - Peter's action: based on the work presented, make a decision about conference
     submission (see below Jan milestone for approval)
Dec:
   - expand existing work from couple of data-providers into full set of CMS  data-services
   - work on integration with existing DAS code and prepare patch for  production deployment
   - plenty of testing, including but not limited to:
     - unit tests
     - integration tests
     - end-user UI studies
           - good to have prototype and give it to end-users
           - study and observe user behavior
   - present final strategy at EPFL & CMS DMWM team, including benchmarks, stats, user UI studies
Jan:
   - work on optimization (if required)
   - integration with DAS and production deployment
   - prepare main sections of write-up document
     - target approach & math foundation
     - prototype description
     - user studies, benchmarks, stats
   - submit write-up for EPFL & CMS review
     - if it is done, Peter will need to get some approval for conference presentation
Feb:
   - final document in place describing work, achievements, results
   - make conclusion about contribution to the field
     - discuss innovation work
     - discuss engineering work
     - discuss CMS use case and achievements
     - discuss possible application outside the field
Mar 15:
    - thesis delivery
Apr:
   - defense
\end{Verbatim}
\end{footnotesize}

\end{appendix}

\pagebreak
%\appendix
\begin{appendix}
\section{DAS Query logs (from 2011-06-21 to 2012-10-01)}

From the logs it can be seen that queries requiring heavy joins are quite common (100K queries through command line interface) that makes it worth investigating the possible performance optimizations.

\subsection{Common query patterns through Web browser}
total valid queries: 569,408
\\
not well formed queries (e.g. free text, typing mistakes, spam): 98,923

\begin{Verbatim}[commandchars=\\\{\},numbers=left,numbersep=4pt]
50.16% (285605)	:	dataset dataset.name=?
13.54% (77071)	:	site dataset.name=?
8.96%  (51035)	:	file dataset.name=?
5.88%  (33504)	:	run dataset.name=?
2.59%  (14739)	:	release dataset.name=?
2.11%  (12036)	:	config dataset.name=?
\textcolor{red}{1.65%  (9420) 	:	dataset run.run_number=?}
1.34%  (7642) 	:	block dataset.name=?
1.24%  (7084) 	:	dataset site.name=?
1.15%  (6562) 	:	dataset dataset.name=? release.name=?
1.10%  (6287) 	:	parent dataset.name=?
0.96%  (5488) 	:	file file.name=?
0.92%  (5245) 	:	dataset dataset.name=? status.name=?
0.77%  (4363) 	:	dataset release.name=?
\textcolor{red}{0.75%  (4257) 	:	file dataset.name=? run.run_number=?}
0.68%  (3874) 	:	run run.run_number=?
0.53%  (2994) 	:	site site.name=?
0.45%  (2576) 	:	site file.name=?
0.45%  (2556) 	:	dataset file.name=?
0.43%  (2438) 	:	lumi file.name=?
\textcolor{red}{0.40%  (2282) 	:	dataset dataset.name=? site.name=?}
0.35%  (1999) 	:	file block.name=?
0.35%  (1970) 	:	dataset dataset.name=? run.run_number=?
0.29%  (1640) 	:	group dataset.name=?
0.29%  (1631) 	:	lumi run.run_number=?
\end{Verbatim}

\begin{verbatim}
Interesting non-valid queries:
* keyword search: *herwig*/AODSIM
* Users may like more complex combined queries:
    lumi file = (file dataset=/RelValProdTTbar/JobRobot-MC_42_V12_JobRobot-v1/GEN-SIM-RECO)
    file,lumi dataset=/RelValProdTTbar/JobRobot-MC_42_V12_JobRobot-v1/GEN-SIM-RECO
* Users mixing up the post and pre filters:
    file dataset=/MuEG/Run2011B-PromptReco-v1/AOD, file.size >1
    file dataset=/MinimumBias/Run2010A-valskim-v6/RAW-RECO* | grep run between  [138923, 144086] 
\end{verbatim}
\newpage
\subsection{Common query patterns through Command Line\label{appendix_das_cli_logs}}
total valid queries: 6,9M\\
non valid queries: 76K

\begin{Verbatim}[commandchars=\\\{\},numbers=left,numbersep=4pt]
39.56% (2735728):	dataset dataset.name=?
22.80% (1576886):	dataset dataset.name=? status.name=?
12.49% (863636) :	file dataset.name=?
12.44% (860626) :	run run.run_number=?
4.60%  (318248) :	site dataset.name=?
2.18%  (150845) :	run dataset.name=?
\textcolor{red}{1.71%  (118404) :	file dataset.name=? run.run_number=?}\label{run_dataset_heavy}
1.32%  (90988)  :	file block.name=?
1.26%  (87266)  :	file file.name=?
0.50%  (34482)  :	block site.name=?
0.35%  (24049)  :	lumi file.name=?
0.25%  (17282)  :	release dataset.name=?
0.18%  (12556)  :	parent file.name=?
0.09%  (6341)   :	file dataset.name=? lumi.number=? run.run_number=?
\textcolor{red}{0.06%  (4306)   :	file dataset.name=? site.name=?}\label{site_dataset_distr_heavy}
0.05%  (3547)   :	dataset dataset.name=? primary_dataset.name=? release.name=? tier.name=?
0.05%  (3352)   :	dataset file.name=?
0.01%  (996)    :	parent dataset.name=?
0.01%  (755)    :	dataset site.name=?
\end{Verbatim}


% 
% Possible keyword search test cases:
% run duration run_number between [195100,195200] 
%
% py dataset=/DYJetsToLL_M-50_TuneZ2Star_8TeV-madgraph-tarball/Summer12_DR53X-PU_S10_START53_V7A-v1/AODSIM
%dataset dataset=*Run2011* datatype=data dataset=*RAW*

\end{appendix}
\begin{appendix}
%\appendix
\section{Data providers statistics}
\begin{verbatim}
(Some of the largest ones)

DBS:
    DB  size: 80GB + 200GB indexes, not many changes to existing  (old) records
    Largest tables: 
      Dataset (164K rows) -> Block (2M) -> Files (31M) -> FileRunLumi (902M) <- Runs (65K)

Phedex: ~7GB, more often changes to existing (even old) records
    change rate: 2,359,934 file transfers last month (from site A to site B; 
                 change rate on the DB to be found out)
\end{verbatim}
\end{appendix}


{\color{gray}
\subsubsection*{Not so relevant from Literature Review: Deep web search at Google}
%[multiple papers from Google] discusses various ways for implementing data integration in terms of large-scale search engine (Google): Virtual integration vs. Surfacing. They also present ways for integrating systems without human intervention through use of statistical 'mediator'.

There are two approaches to web scale search over deep-web (here Google mostly cares about web forms): 
%\marginpar{\scriptsize\color{red}TODO: how does google use PayGo?} 
%\textit{Runtime query reformulation} - 'leaves data at the sources and routes queries to appropriate services'\cite[p. 1]{webscale_paygo} 

\textit{Deep-web surfacing} - surface deep-web (e.g. web forms) adding their results into the standard search index  easily allowing to using existing IR technology that scales well. There exist algorithms which allow to iteratively choose input parameters to the forms to surface a considerable part of the 'hidden' data without large overhead%
	\footnote{i.e. if choosing parameters in not smart way, a web form with just a couple of free inputs (or even dropdowns that's easier case), could yield as many results as a cross product of all input combinations.}.

\textit{Pay-as-you-go approach}\cite{webscale_paygo} - With this approach, 'a system starts with
very few (or inaccurate) semantic mappings and these mappings are improved over time as deemed necessary';
% TODO: cite{bootstraping}
there is NO single mediated schema over which users pose queries: queries are routed to the relevant sources with help statistical methods that are used to model uncertainty at all levels: queries, mappings and underlying data. 
}

\end{document}
