
% Template for Elsevier CRC journal article
% version 1.2 dated 09 May 2011

% This file (c) 2009-2011 Elsevier Ltd.  Modifications may be freely made,
% provided the edited file is saved under a different name

% This file contains modifications for Procedia Computer Science

% Changes since version 1.1
% - added "procedia" option compliant with ecrc.sty version 1.2a
%   (makes the layout approximately the same as the Word CRC template)
% - added example for generating copyright line in abstract

%-----------------------------------------------------------------------------------

%% This template uses the elsarticle.cls document class and the extension package ecrc.sty
%% For full documentation on usage of elsarticle.cls, consult the documentation "elsdoc.pdf"
%% Further resources available at http://www.elsevier.com/latex

%-----------------------------------------------------------------------------------

%%%%%%%%%%%%%%%%%%%%%%%%%%%%%%%%%%%%%%%%%%%%%%%%%%%%%%%%%%%%%%
%%%%%%%%%%%%%%%%%%%%%%%%%%%%%%%%%%%%%%%%%%%%%%%%%%%%%%%%%%%%%%
%%                                                          %%
%% Important note on usage                                  %%
%% -----------------------                                  %%
%% This file should normally be compiled with PDFLaTeX      %%
%% Using standard LaTeX should work but may produce clashes %%
%%                                                          %%
%%%%%%%%%%%%%%%%%%%%%%%%%%%%%%%%%%%%%%%%%%%%%%%%%%%%%%%%%%%%%%
%%%%%%%%%%%%%%%%%%%%%%%%%%%%%%%%%%%%%%%%%%%%%%%%%%%%%%%%%%%%%%

%% The '3p' and 'times' class options of elsarticle are used for Elsevier CRC
%% The 'procedia' option causes ecrc to approximate to the Word template
\documentclass[3p,times,procedia]{elsarticle}

%% The `ecrc' package must be called to make the CRC functionality available
\usepackage{ecrc, fleqn}

%% The ecrc package defines commands needed for running heads and logos.
%% For running heads, you can set the journal name, the volume, the starting page and the authors

%% set the volume if you know. Otherwise `00'
\volume{00}

%% set the starting page if not 1
\firstpage{1}

%% Give the name of the journal
\journalname{Procedia Computer Science}

%% Give the author list to appear in the running head
%% Example \runauth{C.V. Radhakrishnan et al.}
\runauth{Author name}

%% The choice of journal logo is determined by the \jid and \jnltitlelogo commands.
%% A user-supplied logo with the name <\jid>logo.pdf will be inserted if present.
%% e.g. if \jid{yspmi} the system will look for a file yspmilogo.pdf
%% Otherwise the content of \jnltitlelogo will be set between horizontal lines as a default logo

%% Give the abbreviation of the Journal.
\jid{procs}

%% Give a short journal name for the dummy logo (if needed)
\jnltitlelogo{Procedia Computer Science}

%% Provide the copyright line to appear in the abstract
%% Usage:
%   \CopyrightLine[<text-before-year>]{<year>}{<restt-of-the-copyright-text>}
%   \CopyrightLine[Crown copyright]{2011}{Published by Elsevier Ltd.}
%   \CopyrightLine{2011}{Elsevier Ltd. All rights reserved}
%\CopyrightLine{2012}{The Authors. Published by Elsevier B.V.\newline Selection and/or peer-review under responsibility of Scientific Programme Committee of the 4th International Conference on Software Development for Enhancing Accessibility and Fighting Info-exclusion (DSAI 2012)}


%% Hereafter the template follows `elsarticle'.
%% For more details see the existing template files elsarticle-template-harv.tex and elsarticle-template-num.tex.

%% Elsevier CRC generally uses a numbered reference style
%% For this, the conventions of elsarticle-template-num.tex should be followed (included below)
%% If using BibTeX, use the style file elsarticle-num.bst

%% End of ecrc-specific commands
%%%%%%%%%%%%%%%%%%%%%%%%%%%%%%%%%%%%%%%%%%%%%%%%%%%%%%%%%%%%%%%%%%%%%%%%%%

%% The amssymb package provides various useful mathematical symbols
\usepackage{amssymb}
%% The amsthm package provides extended theorem environments
%% \usepackage{amsthm}

%% The lineno packages adds line numbers. Start line numbering with
%% \begin{linenumbers}, end it with \end{linenumbers}. Or switch it on
%% for the whole article with \linenumbers after \end{frontmatter}.
%% \usepackage{lineno}

%% natbib.sty is loaded by default. However, natbib options can be
%% provided with \biboptions{...} command. Following options are
%% valid:

%%   round  -  round parentheses are used (default)
%%   square -  square brackets are used   [option]
%%   curly  -  curly braces are used      {option}
%%   angle  -  angle brackets are used    <option>
%%   semicolon  -  multiple citations separated by semi-colon
%%   colon  - same as semicolon, an earlier confusion
%%   comma  -  separated by comma
%%   numbers-  selects numerical citations
%%   super  -  numerical citations as superscripts
%%   sort   -  sorts multiple citations according to order in ref. list
%%   sort&compress   -  like sort, but also compresses numerical citations
%%   compress - compresses without sorting
%%
%% \biboptions{comma,round}

% \biboptions{}

% if you have landscape tables
\usepackage[figuresright]{rotating}

% put your own definitions here:
%   \newcommand{\cZ}{\cal{Z}}
%   \newtheorem{def}{Definition}[section]
%   ...

% add words to TeX's hyphenation exception list
%\hyphenation{author another created financial paper re-commend-ed Post-Script}

% declarations for front matter

\begin{document}

\begin{frontmatter}

%% Title, authors and addresses

%% use the tnoteref command within \title for footnotes;
%% use the tnotetext command for the associated footnote;
%% use the fnref command within \author or \address for footnotes;
%% use the fntext command for the associated footnote;
%% use the corref command within \author for corresponding author footnotes;
%% use the cortext command for the associated footnote;
%% use the ead command for the email address,
%% and the form \ead[url] for the home page:
%%
%% \title{Title\tnoteref{label1}}
%% \tnotetext[label1]{}
%% \author{Name\corref{cor1}\fnref{label2}}
%% \ead{email address}
%% \ead[url]{home page}
%% \fntext[label2]{}
%% \cortext[cor1]{}
%% \address{Address\fnref{label3}}
%% \fntext[label3]{}

\dochead{International Conference on Computational Science, ICCS 2013}
%% Use \dochead if there is an article header, e.g. \dochead{Short communication}
%% \dochead can also be used to include a conference title, if directed by the editors
%% e.g. \dochead{17th International Conference on Dynamical Processes in Excited States of Solids}

\title{Click here, type the title of your paper, Capitalize first letter}

%% use optional labels to link authors explicitly to addresses:
%% \author[label1,label2]{<author name>}
%% \address[label1]{<address>}
%% \address[label2]{<address>}

\author[a]{First Author} 
\author[b]{Second Authorb}
\author[a,b]{Third Author\corref{cor1}}

\address[a]{First affiliation, Address, City and Postcode, Country}
\address[b]{Second affiliation, Address, City and Postcode, Country}

\begin{abstract}
%% Text of abstract
Click here and insert your abstract text.\vskip11pt
\end{abstract}

\begin{keyword}
Type your keywords here, separated by semicolons ; 

%% keywords here, in the form: keyword \sep keyword

%% PACS codes here, in the form: \PACS code \sep code

%% MSC codes here, in the form: \MSC code \sep code
%% or \MSC[2008] code \sep code (2000 is the default)

\end{keyword}
 \cortext[cor1]{Corresponding author. Tel.: +0-000-000-0000 ; fax: +0-000-000-0000 .}
\end{frontmatter}

%\correspondingauthor[*]{Corresponding author. Tel.: +0-000-000-0000 ; fax: +0-000-000-0000 .}
\email{author@institute.xxx}

%%
%% Start line numbering here if you want
%%
% \linenumbers

%% main text
\section{Main Text}
\label{main}

(10 pt) Here introduce the paper, and put a nomenclature if necessary, in a box with the same font size as the rest of the paper. The paragraphs continue from here and are only separated by headings, subheadings, images and formulae.  The section headings are arranged by numbers, bold and 10 pt.  Here follows further instructions for authors.


\subsection{ Structure}
Files should be in MS Word format only and should be formatted for direct printing. Figures and tables should be embedded and not supplied separately. Please make sure that you use as much as possible normal fonts in your documents. Special fonts, such as fonts used in the Far East (Japanese, Chinese, Korean, etc.) may cause problems during processing. To avoid unnecessary errors you are strongly advised to use the ‘spellchecker’ function of MS Word. Follow this order when typing manuscripts: Title, Authors, Affiliations, Abstract, Keywords, Main text (including figures and tables), Acknowledgements, References, Appendix. Collate acknowledgements in a separate section at the end of the article and do not include them on the title page, as a footnote to the title or otherwise. 

Bulleted lists may be included and should look like this:
\begin{itemize}[]
\item First point
\item Second point
\item And so on
\end{itemize}

Ensure that you return to the `Els-body-text' style, the style that you will mainly be using for large blocks of text, when you have completed your bulleted list.

Please do not alter the formatting and style layouts which have been set up in this template document. As indicated in the template, Papers should be prepared in single column format suitable for direct printing onto paper size (192 x 262) mm. Do not number pages on the front, as page numbers will be added separately for the preprints and the Proceedings. Leave a line clear between paragraphs. All the required style templates are provided in this document with the appropriate name supplied, e.g. choose 1. Els1st-order-head for your first order heading text, els-abstract-text for the abstract text etc.

\subsection{ Tables}

All tables should be numbered with Arabic numerals. Headings should be placed above tables, left justified. Leave one line space between the heading and the table. Only horizontal lines should be used within a table, to distinguish the column headings from the body of the table, and immediately above and below the table. Tables must be embedded into the text and not supplied separately. Below is an example which authors may find useful.

\begin{table}[h]
\caption{An example of a table.}
\begin{tabular*}{\hsize}{@{\extracolsep{\fill}}lll@{}}
\hline
An example of a column heading & Column A (t) & Column B (T)\\
\hline
And an entry &   1 &  2\\
And another entry  & 3 &  4\\
And another entry &  5 &  6\\
\hline
\end{tabular*}
\end{table}


\subsection{Construction of references}

References should be listed at the end of the paper, and numbered in the order of their appearance in the text. Authors should ensure that every reference in the text appears in the list of references and vice versa. Indicate references by numbers in the text. In the text the number of the reference should be given in square brackets [??]. The actual authors can be referred to, but the reference number(s) must always be given.

Some examples of how your references should be listed are given at the end of this template in the ‘References’ section which will allow you to assemble your reference list according to the correct format and font size. There is a shortened form for last page number. e.g., 51–9, and that for more than 6 authors the first 6 should be listed followed by “et al.”


\subsection{Section headings}

Section headings should be left justified, with the first letter capitalized and numbered consecutively, starting with the Introduction. Sub-section headings should be in capital and lower-case italic letters, numbered 1.1, 1.2, etc, and left justified, with second and subsequent lines indented.  You may need to insert a page break to keep a heading with its text.


\subsection{General guidelines for the preparation of your text}
Avoid hyphenation at the end of a line. Symbols denoting vectors and matrices should be indicated in bold type. Scalar variable names should normally be expressed using italics. Weights and measures should be expressed in SI units. Please title your files in this order conferenceacrynom\_authorslastname.pdf


\subsection{Footnote}
Footnotes should be avoided if possible. Necessary footnotes should be denoted in the text by consecutive superscript letters. The footnotes should be typed single spaced, and in smaller type size (8pt), at the foot of the page in which they are mentioned, and separated from the main text by a short line extending at the foot of the column.  The ‘Els-footnote’ style is available in this template for the text of the footnote.

\section{ Author Artwork}
All figures should be numbered with Arabic numerals (1,2,...n). All photographs, schemas, graphs and diagrams are to be referred to as figures. Line drawings should be good quality scans or true electronic output. Low-quality scans are not acceptable. Figures must be embedded into the text and not supplied separately. Lettering and symbols should be clearly defined either in the caption or in a legend provided as part of the figure. Figures should be placed at the top or bottom of a page wherever possible, as close as possible to the first reference to them in the paper.

The figure number and caption should be typed below the illustration in 9pt and left justified. For more guidelines and information to help you submit high quality artwork please visit: http://www.elsevier.com/wps/find/ authorsview.authors/authorartworkinstructions. Artwork has no text along the side of it in the main body of the text.  However, if two images fit next to each other, these may be placed next to each other to save space, see Fig 1. They must be numbered consecutively, all figures, and all tables respectively.

\begin{figure}[h]
\centerline{\includegraphics{fx1}\hspace*{5mm}\includegraphics{fx1}}
\caption{(a) first picture; (b) second picture.}
\end{figure}


Equations and formulae should be typed and numbered consecutively with Arabic numerals in parentheses on the right hand side of the page (if referred to explicitly in the text),
\begin{equation}
Rt = K EP = 93.02 (\pm 9.62) – 13.45
\end{equation}

They should also be separated from the surrounding text by one space.

\section{Copyright}
All authors must sign the Transfer of Copyright agreement before the article can be published. This transfer agreement enables Elsevier to protect the copyrighted material for the authors, but does not relinquish the authors' proprietary rights. The copyright transfer covers the exclusive rights to reproduce and distribute the article, including reprints, photographic reproductions, microfilm or any other reproductions of similar nature and translations. Authors are responsible for obtaining from the copyright holder permission to reproduce any figures for which copyright exists.


The citation must be used in following style: \cite{article-minimal}, \cite{article-full}, \cite{article-crossref}, \cite{whole-journal} and \cite{inbook-minimal}.

\section*{Acknowledgements}

These and the Reference headings are in bold but have no numbers. Text below continues as normal. 

%% References
%%
%% Following citation commands can be used in the body text:
%% Usage of \cite is as follows:
%%   \cite{key}         ==>>  [#]
%%   \cite[chap. 2]{key} ==>> [#, chap. 2]
%%


%% References with BibTeX database:

\bibliographystyle{elsarticle-num}
\bibliography{xampl}

%% Authors are advised to use a BibTeX database file for their reference list.
%% The provided style file elsarticle-num.bst formats references in the required Procedia style

%% For references without a BibTeX database:

% \begin{thebibliography}{00}

%% \bibitem must have the following form:
%%   \bibitem{key}...
%%

% \bibitem{}

% \end{thebibliography}


%% The Appendices part is started with the command \appendix;
%% appendix sections are then done as normal sections
%% \appendix

%% \section{}
%% \label{}

\appendix
\section{An example appendix}
Authors including an appendix section should do so after References section. Multiple appendices should all have headings in the style used above. They will automatically be ordered A, B, C etc.

\subsection{Example of a sub-heading within an appendix}
There is also the option to include a subheading within the Appendix if you wish.

\end{document}

%%
%% End of file `procs-template.tex'.
