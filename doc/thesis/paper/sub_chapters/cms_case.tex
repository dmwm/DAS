At the start of this project, an enterprise information integration system based on simple structured queries was already in place (described below), which suffered from a number of performance and usability problems. %
The author of this work evaluated possible approaches to these problems: improvements to system's usability through redesigning the user interface and allowing less restricted keyword search, analysis of performance bottlenecks {\color{red}and the possibility of supporting more complex queries}.\marginpar{{\color{red}only to some extent - secondary goal. we are bound by performance issues first}}

% {\color{red}A remarkable opportunity we have is possibility to test ideas on a sample of ~3000 people working at CMS.}


%Following a similar approach a 
\subsubsection*{The EII system used at CMS, CERN}

The \textit{CMS Data Aggregation System (DAS)}\cite{Kuznetsov2010, Kuznetsov2011} allows integrated access to a number of proprietary data-sources by processing users' queries on demand - it determines which  data-sources are able to answer\footnote{This is done by a mapping between flat mediated schema entities ('DAS keys') into web-service methods and their arguments. Then system queries all services that could provide a result of expected type with given parameters}, queries them, merges the results and caches them for subsequent use. DAS uses the \textit{Boolean retrieval model} as users are often interested in retrieving ALL the items matching their query.

Currently the queries are formed by specifying what entity the user is interested in (dataset, file, etc) and providing selection criteria (e.g. attribute=value, name BETWEEN [v1, v2]). The combined query results could be later 'piped' for further filtering, sorting or aggregation (min, max, avg, sum, count, median), e.g.:

{\small 
\begin{Verbatim}[commandchars=\\\{\}]
\footnotesize# find average and median dataset sizes of \textit{RelVal} datasets with more than 1000 events
dataset=*RelVal* | grep dataset.nevents >1000 | avg(dataset.size), median(dataset.size)
\end{Verbatim}
}

Queries are executed either from web browser or through a command line interface where the results could be fed into another application (e.g. programs doing physics analysis or automatic software release validation).
