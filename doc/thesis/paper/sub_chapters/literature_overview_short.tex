% Overview: keyword search is good for non-structured documents, it is not [as] effective with structured sources\cite{Levy96}, therefore the keyword search on relational databases approach is good then no specific APIs exist and it's there are no resources to do \textbf{a full integration} .

%\section{Overview}

% Data Integration is about balancing between Virtual integration and data warehousing.

Since the late 1990's, several \textit{Enterprise Information Integration}\footnote{%
	Enterprise Information Integration (EII) is about 'integrating data from 
	multiple sources 	\textit{without} having to to first load data into
	 a central warehouse'\cite[p.1]{eii_2005}}
 (EII) products have appeared in the market (e.g. \textit{Information Manifold} by AT\&T Lab) and an significant experience has been accumulated on data integration formalisms, ways of describing heterogeneous data sources and their abilities (e.g. RDBMS vs web form), query optimization (combining sources efficiently, source overlap, data quality, etc)\cite{eii_2005}. 
%
Recent research in  Enterprise Information Integration mostly focused on approaches minimizing  human efforts on source integration, e.g. on uncertainty based self-improving systems\cite[ch.19]{principles_data_integration}.\marginpar{\color{red} cite from survey: \cite{IIHet_survey08}}

The problem of keyword search over structured sources received a significant attention within the last decade. Keyword search over relational and other structured databases was explored from a number of perspectives: returning top-k ranked result tuples vs suggesting structured queries as SQL, performance optimization, user feedback mechanisms, keyword searching over distributed sources, up to lightweight exploratory\footnote{because of probabilistic nature of schema mappings, it do not provide 100\% result exactness} probabilistic data integration based on users-feedback minimizing the upfront human effort required\cite[ch.16]{principles_data_integration}. 
%
%If there is no need for 100\% result exactness, keyword search combined with probabilistic schema matching provides lightweight exploratory data integration with almost no human effort upfront with ability to self-improve given users' feedback\cite[ch.16]{principles_data_integration}. 
%
On the other extreme, the \textit{SODA}{ethz2012} system has proved that if enough meta-data is in place, even quite complex queries given in bussiness terms could be answered over a large and complex warehouse.

Meanwhile, to the best of our knowledge, the \textit{Boolean Keyword-based search over heterogeneous sources}, which is our main focus, received much less attention from the research community. 
%
{\color{magenta}As the users are interested in complete answer sets, the standard approach is translating the keywords into a ranked list of structured queries (e.g.\textit{SODA} system that propose SQL over large datawarehouse, or KEYRY/Keymantic trying to achieve this without accessing the data).}
%
We have identified these three categories of related research contributions: heuristic based-keyword search, machine learning-based keyword search, and {\color{magenta}less mature works} attemping to process multi-domain natural language queries.
%
%\marginpar{\color{red}TODO: describe keyword search}
%In the extreme case of having no control over a web-service that do not publish its contents, techniques like Google's Deep-web surfacing could index a subset of its contents enabling keyword search to some extent.
% \textit{Query Forms} approach that propose SQL templates
